\documentclass[12pt,a4paper,oneside]{article}

\usepackage[margin=3cm]{geometry}

\usepackage{hyperref}
\hypersetup{
    pdftitle={COM 299, Game Development},%
    pdfauthor={Toksaitov Dmitrii Alexandrovich},%
    pdfsubject={Syllabus},%
    pdfkeywords={COM;}{299;}{syllabus;}{game;}{development},%
    colorlinks,%
    linkcolor=black,%
    citecolor=black,%
    filecolor=black,%
    urlcolor=black
}

\newcommand{\R}[1]{\uppercase\expandafter{\romannumeral #1\relax}}

\begin{document}

    \title{COM 299, Game Development}
    \author{
        American University of Central Asia\\
        Department of Software Engineering
    }
    \date{}
    \maketitle

    \section{Course Information}

        \begin{description}
            \item[Course ID]\hfill\\
                COM 299, 3956
            \item[Course Repository]\hfill\\
                \url{https://github.com/auca/com.299}
            \item[Class Discussions]\hfill\\
                \url{https://piazza.com/class/j6teu837xx66jh}
            \item[Place]\hfill\\
                AUCA, laboratory G31
            \item[Time]\hfill\\
                Monday 10:50\\
                Friday 10:50
        \end{description}

    \section{Prerequisites}

        COM 117, Programming \R{2}. Object-oriented Design

    \section{Contact Information}

        \begin{description}
            \item[Instructor]\hfill\\
                Toksaitov Dmitrii Alexandrovich\\
                \href{mailto:toksaitov_d@auca.kg}{toksaitov\_d@auca.kg}
            \item[Office]\hfill\\
                AUCA, room 315
            \item[Office Hours]\hfill\\
                Monday 15:25--17:00\\
                Tuesday 15:25--17:00\\
                Wednesday 10:00--17:00\\
                Thursday 15:25--17:00\\
                Friday 15:25--17:00
        \end{description}

    \section{Course Overview}

        The course introduces students to the topic of game development. It
        covers theory and practice of video game production. It delves into the
        fields of computer graphics, computational physics, artificial
        intelligence, and game-play design. During the course students will get
        an opportunity to build a market-ready game for desktop, web, or mobile
        platforms.  Students will learn on how to use the Unity game engine, the
        leading game authoring tool on the market. Students will also take a
        look on a popular alternative, Unreal Engine 4. Finally, they will be
        introduced to the topic of building a simple game engine from scratch on
        their own.

    \section{Examinations}

        Students will get midterm and final examinations in the form of two
        quizzes with multiple choice questions. Both examinations will be about
        linear algebra topics for use in game development.

    \section{Practice Tasks}

        Students will have to finish several practice tasks. In each task they
        will have to create simple clones of classic video games from various
        genres.

    \section{Course Project}

        Throughout the course, students will have to create a game on their own
        or together with another student in a team. It is up to the student or
        the team to select the type of the game to make.

    \section{Presentation}

        Students will have to make one presentation about any market game of
        their choice. The presentation should be focused on the game's
        internals, its development or production process, and tools or
        techniques used to create it.

    \section{Reading}

        3D Math Primer for Graphics and Game Development, Second Edition by
        Fletcher Done and Ian Parberry (ISBN: 978-1-4398-6981-9)

        \subsection{Supplemental Reading}

            \begin{enumerate}
                \item Game Development Essentials: An Introduction 3rd Edition
                by Jeannie Novak (ISBN: 978-1111307653)
                \item Game Coding Complete, Fourth Edition by McShaffry and
                David Graham (ISBN: 978-1133776574)
                \item Game Engine Architecture, Second Edition by Jason Gregory
                (ISBN: 978-1568814131)
                \item Game Programming Patterns by Robert Nystrom (ISBN:
                978-0990582908)
                \item Mathematics for 3D Game Programming and Computer Graphics,
                Third Edition by by Eric Lengyel (ISBN: 978-1435458864)
            \end{enumerate}

    \section{Grading}

        \begin{itemize}
            \item Class participation (through Piazza) (5\%)
            \item Presentation (10\%)
            \item Midterm (15\%)
            \item Final (20\%)
            \item Practice tasks (20\%)
            \item The course project (30\%)
        \end{itemize}

        \begin{itemize} \itemsep-10pt \parskip0pt \parsep0pt
            \item[--] 90\%--100\%: A\\
            \item[--] 80\%--89\%: A-\\
            \item[--] 70\%--79\%: B+\\
            \item[--] 65\%--69\%: B\\
            \item[--] 60\%--64\%: B-\\
            \item[--] 56\%--59\%: C+\\
            \item[--] 53\%--55\%: C\\
            \item[--] 50\%--52\%: C-\\
            \item[--] 46\%--49\%: D+\\
            \item[--] 43\%--45\%: D\\
            \item[--] 40\%--42\%: D-\\
            \item[--] Less than 39\%: F
        \end{itemize}

    \section{Rules}

        Students are required to follow the rules of conduct of the Software
        Engineering Department and American University of Central Asia.

        Team work is NOT encouraged. The same blocks of code or similar
        structural pieces in separate works will be considered as academic
        dishonesty and all parties will get zero for the task.

\end{document}


